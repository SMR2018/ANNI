% Options for packages loaded elsewhere
\PassOptionsToPackage{unicode}{hyperref}
\PassOptionsToPackage{hyphens}{url}
%
\documentclass[
]{article}
\usepackage{amsmath,amssymb}
\usepackage{lmodern}
\usepackage{iftex}
\ifPDFTeX
  \usepackage[T1]{fontenc}
  \usepackage[utf8]{inputenc}
  \usepackage{textcomp} % provide euro and other symbols
\else % if luatex or xetex
  \usepackage{unicode-math}
  \defaultfontfeatures{Scale=MatchLowercase}
  \defaultfontfeatures[\rmfamily]{Ligatures=TeX,Scale=1}
\fi
% Use upquote if available, for straight quotes in verbatim environments
\IfFileExists{upquote.sty}{\usepackage{upquote}}{}
\IfFileExists{microtype.sty}{% use microtype if available
  \usepackage[]{microtype}
  \UseMicrotypeSet[protrusion]{basicmath} % disable protrusion for tt fonts
}{}
\makeatletter
\@ifundefined{KOMAClassName}{% if non-KOMA class
  \IfFileExists{parskip.sty}{%
    \usepackage{parskip}
  }{% else
    \setlength{\parindent}{0pt}
    \setlength{\parskip}{6pt plus 2pt minus 1pt}}
}{% if KOMA class
  \KOMAoptions{parskip=half}}
\makeatother
\usepackage{xcolor}
\usepackage[margin=1in]{geometry}
\usepackage{listings}
\newcommand{\passthrough}[1]{#1}
\lstset{defaultdialect=[5.3]Lua}
\lstset{defaultdialect=[x86masm]Assembler}
\usepackage{graphicx}
\makeatletter
\def\maxwidth{\ifdim\Gin@nat@width>\linewidth\linewidth\else\Gin@nat@width\fi}
\def\maxheight{\ifdim\Gin@nat@height>\textheight\textheight\else\Gin@nat@height\fi}
\makeatother
% Scale images if necessary, so that they will not overflow the page
% margins by default, and it is still possible to overwrite the defaults
% using explicit options in \includegraphics[width, height, ...]{}
\setkeys{Gin}{width=\maxwidth,height=\maxheight,keepaspectratio}
% Set default figure placement to htbp
\makeatletter
\def\fps@figure{htbp}
\makeatother
\setlength{\emergencystretch}{3em} % prevent overfull lines
\providecommand{\tightlist}{%
  \setlength{\itemsep}{0pt}\setlength{\parskip}{0pt}}
\setcounter{secnumdepth}{-\maxdimen} % remove section numbering
\ifLuaTeX
\usepackage[bidi=basic]{babel}
\else
\usepackage[bidi=default]{babel}
\fi
\babelprovide[main,import]{ngerman}
% get rid of language-specific shorthands (see #6817):
\let\LanguageShortHands\languageshorthands
\def\languageshorthands#1{}
\lstset{
  breaklines=true
}
\usepackage{float}
\ifLuaTeX
  \usepackage{selnolig}  % disable illegal ligatures
\fi
\IfFileExists{bookmark.sty}{\usepackage{bookmark}}{\usepackage{hyperref}}
\IfFileExists{xurl.sty}{\usepackage{xurl}}{} % add URL line breaks if available
\urlstyle{same} % disable monospaced font for URLs
\hypersetup{
  pdftitle={ANNI anwenden: Paxisphase 2023},
  pdfauthor={Samantha Rubo},
  pdflang={de},
  hidelinks,
  pdfcreator={LaTeX via pandoc}}

\title{ANNI anwenden: Paxisphase 2023}
\author{Samantha Rubo}
\date{2023-04-20}

\begin{document}
\maketitle

\hypertarget{vorgehen}{%
\subsection{Vorgehen:}\label{vorgehen}}

\hypertarget{daten-und-modell-laden}{%
\subsubsection{1. Daten und Modell laden}\label{daten-und-modell-laden}}

\hypertarget{gespiechertes-modell-laden}{%
\subsubsection{2. Gespiechertes Modell
laden}\label{gespiechertes-modell-laden}}

\hypertarget{skalierungs-werte-fuxfcr-input-daten-laden}{%
\subsubsection{3. Skalierungs-Werte für Input-Daten
laden}\label{skalierungs-werte-fuxfcr-input-daten-laden}}

\hypertarget{wetter--und-anbaudaten-einlesen}{%
\subsubsection{4. Wetter- und Anbaudaten
einlesen}\label{wetter--und-anbaudaten-einlesen}}

\hypertarget{anni-fuxfcr-jeden-tag-einzeln-hintereinander-im-loop-ausfuxfchren}{%
\subsubsection{5. ANNI Für jeden Tag einzeln hintereinander im Loop
ausführen}\label{anni-fuxfcr-jeden-tag-einzeln-hintereinander-im-loop-ausfuxfchren}}

\hypertarget{ergebnisse-formatieren}{%
\subsubsection{6. Ergebnisse formatieren}\label{ergebnisse-formatieren}}

\hypertarget{anni-ergebnis-plotten}{%
\subsubsection{7. ANNI-Ergebnis plotten}\label{anni-ergebnis-plotten}}

\hfill\break

\begin{figure}[H]
\includegraphics{ANNI_anwenden_Praxis_files/figure-latex/unnamed-chunk-9-1} \caption{Ergebnis des Bewässerungsmodells ANNI für die Fläche Rosengarten der Fa. Haas für die Tage 0-50 seit Aussaat. Dargestellt sind A) Wetterdaten für Niederschlag (mm) und mittlerer Tagestemperatur (°C) und B) die modellierte nutzbare Feldkapazität (nFK \%) für 6 Bodentiefen von 0-60 cm Tiefe.}\label{fig:unnamed-chunk-9}
\end{figure}

\hfill\break

\hfill\break

\begin{lstlisting}
## Grafik gespeichert unter: ../../graphics/ANNI_Praxis/ANNI_Ergebnis_Haas_Rosengarten_Satz_1_DAS0_50.png
\end{lstlisting}


\end{document}
